\documentclass[10pt,a4paper]{article}

\usepackage{lmodern} % Modern font
\usepackage[
    left=1.5cm, right=1.5cm, top=2cm, bottom=2cm
]{geometry}

\usepackage{tcolorbox}
\tcbuselibrary{skins} % Enhanced styling for tcolorbox
\usepackage{booktabs} % Improved table rules
\usepackage{xcolor,colortbl} % Alternating row colors
\usepackage{siunitx} % SI units
\usepackage{fdsymbol} % Symbols
\usepackage{wasysym} % Moon symbol
\usepackage{float}
\usepackage{tabularx}
\usepackage{calc}

\usepackage[compact]{titlesec}
\titlespacing*{\section} {0pt}{2.5ex plus 1ex minus .2ex}{1.3ex plus .2ex}
\titleformat{\section}[block]{\color{black}\Large\it\filcenter}{}{1em}{}

\usepackage{titling}
\usepackage{fancyhdr}

\begin{document}
\title{Constantes fondamentales et ordres de grandeur}

\pagestyle{fancy}
\fancyhead{}

\fancyhead[L]{\scshape Lycée Louis le Grand}
\fancyhead[C]{Constantes et ordres de grandeur}
\fancyhead[R]{\scshape Année 2024-2025}

\author{}
\date{}

\section*{Constantes fondamentales}

\begin{table}[H]
    \centering
    \renewcommand{\arraystretch}{1.3} % Espacement des lignes
    \setlength{\tabcolsep}{10pt} % Espacement des colonnes
    \arrayrulecolor{gray} % Couleur des lignes de tableau
    \rowcolors{2}{gray!15}{white} % Alternance des couleurs
    \caption{Principales constantes physiques}
    \begin{tabularx}{\linewidth}{@{} X r X @{}}
        \toprule
        \textbf{Nom} & \textbf{Valeur} & \textbf{Unité} \\
        \midrule
        Constante de gravitation & \num{6.67e-11} & $\unit{\newton\meter\squared\per\kilogram\squared}$ \\
        Vitesse de la lumière & \num{3.00e8} & $\unit{\meter\per\second}$ \\
        Constante de Planck & \num{6.63e-34} & $\unit{\joule\second}$ \\
        Constante de Boltzmann & \num{1.38e-23} & $\unit{\joule\per\kelvin}$ \\
        Permittivité du vide & \num{8.85e-12} & $\unit{\farad\per\meter}$ \\
        Perméabilité du vide & $4\pi\times 10^{-7}$ & $\unit{\henry\per\meter}$ \\
        Champ de claquage de l'air sec & \num{10e5} & $\unit{\volt\per\meter}$ \\
        Constante de Stefan-Boltzmann & \num{5.67e-8} & $\unit{\watt\per\meter\squared\per\kelvin^4}$ \\
        Constante d'Avogadro & \num{6.022e23} & $\unit{\per\mole}$ \\
        Constante des gaz parfaits & \num{8.31} & $\unit{\joule\per\mole\per\kelvin}$ \\
        Coefficient de Laplace $\gamma$ & $\gamma \simeq 1.4$ & \\
        K standard, autoprotolyse de l'eau & \num{10e-14} & \\
        \bottomrule
    \end{tabularx}
\end{table}

\section*{Propriétés des particules élémentaires}
\begin{table}[H]
    \centering
    \renewcommand{\arraystretch}{1.3} % Espacement des lignes
    \setlength{\tabcolsep}{10pt} % Espacement des colonnes
    \arrayrulecolor{gray} % Couleur des lignes de tableau
    \rowcolors{2}{gray!15}{white} % Alternance des couleurs
    \caption{Caractéristiques des particules élémentaires}
    \begin{tabularx}{\linewidth-4ex}{@{} X r X @{}}
        \toprule
        \textbf{Nom} & \textbf{Valeur} & \textbf{Unité} \\
        \midrule
        Charge élémentaire & \num{1.60e-19} & $\unit{\coulomb}$ \\
        Masse du proton & \num{1.67e-27} & $\unit{\kilogram}$ \\
        Masse du neutron & \num{1.68e-27} & $\unit{\kilogram}$ \\
        Masse de l'électron & \num{9.11e-31} & $\unit{\kilogram}$ \\
        Rayon du proton & \num{0.84e-15} & $\unit{\meter}$ \\
        Rayon du neutron & \num{0.8e-15} & $\unit{\meter}$ \\
        Rayon de l'électron & \num{2.82e-15} & $\unit{\meter}$ \\  
        \bottomrule
    \end{tabularx}
\end{table}

\section*{Paramètres astronomiques}
\begin{table}[H]
    \centering
    \renewcommand{\arraystretch}{1.3} % Espacement des lignes
    \setlength{\tabcolsep}{10pt} % Espacement des colonnes
    \arrayrulecolor{gray} % Couleur des lignes de tableau
    \rowcolors{2}{gray!15}{white} % Alternance des couleurs
    \caption{Caractéristiques astronomiques principales}
    \begin{tabularx}{\linewidth-4ex}{@{} X r X @{}}
        \toprule
        \textbf{Nom} & \textbf{Valeur} & \textbf{Unité} \\
        \midrule
        Constante de gravitation & \num{6.67e-11} & $\unit{\newton\meter\squared\per\kilogram\squared}$ \\
        Masse de la Terre & \num{5.97e24} & $\unit{\kilogram}$ \\
        Rayon moyen de la Terre & \num{6.37e6} & $\unit{\meter}$ \\
        Masse du Soleil & \num{1.989e30} & $\unit{\kilogram}$ \\
        Rayon moyen du Soleil & \num{6.96e8} & $\unit{\meter}$ \\
        Masse de la Lune & \num{7.35e22} & $\unit{\kilogram}$ \\
        Rayon moyen de la Lune & \num{1.74e6} & $\unit{\meter}$ \\
        \bottomrule
    \end{tabularx}
\end{table}

\end{document}
