\documentclass[10pt,a4paper]{article}

\usepackage{lmodern} % Modern font
\usepackage[
    left=1.5cm, right=1.5cm, top=2cm, bottom=2cm
]{geometry}

\usepackage{tcolorbox}
\tcbuselibrary{skins} % Enhanced styling for tcolorbox
\usepackage{booktabs} % Improved table rules
\usepackage{xcolor,colortbl} % Alternating row colors
\usepackage{siunitx} % SI units
\usepackage{fdsymbol} % Symbols
\usepackage{wasysym} % Moon symbol
\usepackage{float}
\usepackage{tabularx}
\usepackage{calc}

\usepackage{ocgx2} %implements PDF Layers
\usepackage{hyperref}

\usepackage[compact]{titlesec}
\titlespacing*{\section} {0pt}{2.5ex plus 1ex minus .2ex}{1.3ex plus .2ex}
\titleformat{\section}[block]{\color{black}\Large\it\filcenter}{}{1em}{}

\usepackage{titling}
\usepackage{fancyhdr}

\let\oldunit\unit
\renewcommand{\unit}[1]
{
    \begin{ocg}{Units}{units}{on}
        \oldunit{#1}
    \end{ocg}
}

\let\oldnum\num
\renewcommand{\num}[1]
{
    \begin{ocg}{Values}{vals}{on}
        \oldnum{#1}
    \end{ocg}
}


\begin{document}
Montrer/Cacher les \switchocg{vals}{valeurs}/ \switchocg{units}{unités} 
    
\title{Constantes fondamentales et ordres de grandeur}

\pagestyle{fancy}
\fancyhead{}

\fancyhead[L]{\scshape Lycée Louis le Grand}
\fancyhead[C]{Constantes et ordres de grandeur}
\fancyhead[R]{\scshape Année 2024-2025}

\author{}
\date{}
\begin{table}[H]
    \centering
    \renewcommand{\arraystretch}{1.3} % Espacement des lignes
    \setlength{\tabcolsep}{10pt} % Espacement des colonnes
    \arrayrulecolor{gray} % Couleur des lignes de tableau
    \rowcolors{2}{gray!15}{white} % Alternance des couleurs
    \caption{Principales constantes physiques}
    \begin{tabularx}{\linewidth}{@{} X r r X @{}}
        \toprule
        \textbf{Nom} & \textbf{Symbole} & \textbf{Valeur} & \textbf{Unité} \\
        \midrule
        Vitesse de la lumière & $c$ & \num{3.00e8} & $\unit{\meter\per\second}$ \\
        Constante de Planck & $h$ & \num{6.63e-34} & $\unit{\joule\second}$ \\
        Constante de Boltzmann & $k_B$ & \num{1.38e-23} & $\unit{\joule\per\kelvin}$ \\
        Permittivité du vide & $\varepsilon_0$ & \num{8.85e-12} & $\unit{\farad\per\meter}$ \\
        Perméabilité du vide & $\mu_0 $& $4\pi\times 10^{-7}$ & $\unit{\henry\per\meter}$ \\
        Champ de claquage de l'air sec & & \num{10e5} & $\unit{\volt\per\meter}$ \\
        Constante de Stefan-Boltzmann & $\sigma$ & \num{5.67e-8} & $\unit{\watt\per\meter\squared\per\kelvin^4}$ \\
        Constante d'Avogadro & $N_A$ & \num{6.022e23} & $\unit{\per\mole}$ \\
        Constante des gaz parfaits & $R$ & \num{8.31} & $\unit{\joule\per\mole\per\kelvin}$ \\
        Coefficient de Laplace & $\gamma$& $\gamma \simeq 1.4$ & \\
        K standard, autoprotolyse de l'eau & & \num{10e-14} & \\
        \bottomrule
    \end{tabularx}
\end{table}

\begin{table}[H]
    \centering
    \renewcommand{\arraystretch}{1.3} % Espacement des lignes
    \setlength{\tabcolsep}{10pt} % Espacement des colonnes
    \arrayrulecolor{gray} % Couleur des lignes de tableau
    \rowcolors{2}{gray!15}{white} % Alternance des couleurs
    \caption{Caractéristiques des particules élémentaires}
    \begin{tabularx}{\linewidth-4ex}{@{} X r r X @{}}
        \toprule
        \textbf{Nom} & \textbf{Symbole} & \textbf{Valeur} & \textbf{Unité} \\
        \midrule
        Charge élémentaire & e & \num{1.60e-19} & $\unit{\coulomb}$ \\
        Masse du proton && \num{1.67e-27} & $\unit{\kilogram}$ \\
        Masse du neutron && \num{1.68e-27} & $\unit{\kilogram}$ \\
        Masse de l'électron &$m_e$& \num{9.11e-31} & $\unit{\kilogram}$ \\
        Rayon du proton && \num{0.84e-15} & $\unit{\meter}$ \\
        Rayon du neutron && \num{0.8e-15} & $\unit{\meter}$ \\
        Rayon de l'électron && \num{2.82e-15} & $\unit{\meter}$ \\  
        \bottomrule
    \end{tabularx}
\end{table}

\begin{table}[H]
    \centering
    \renewcommand{\arraystretch}{1.3} % Espacement des lignes
    \setlength{\tabcolsep}{10pt} % Espacement des colonnes
    \arrayrulecolor{gray} % Couleur des lignes de tableau
    \rowcolors{2}{gray!15}{white} % Alternance des couleurs
    \caption{Caractéristiques astronomiques principales}
    \begin{tabularx}{\linewidth-4ex}{@{} X r r X @{}}
        \toprule
        \textbf{Nom} & \textbf{Symbole} & \textbf{Valeur} & \textbf{Unité} \\
        \midrule
        Constante de gravitation &G& \num{6.67e-11} & $\unit{\newton\meter\squared\per\kilogram\squared}$ \\
        Masse de la Terre && \num{5.97e24} & $\unit{\kilogram}$ \\
        Rayon moyen de la Terre && \num{6.37e6} & $\unit{\meter}$ \\
        Masse du Soleil && \num{1.989e30} & $\unit{\kilogram}$ \\
        Rayon moyen du Soleil && \num{6.96e8} & $\unit{\meter}$ \\
        Masse de la Lune && \num{7.35e22} & $\unit{\kilogram}$ \\
        Rayon moyen de la Lune && \num{1.74e6} & $\unit{\meter}$ \\
        Distance Terre-lune && \num{3.8e8} & $\unit{m}$ \\
        Distance Terre-soleil && \num{1.5e11} & $\unit{m}$ \\
        \bottomrule
    \end{tabularx}
\end{table}

\newpage
Montrer/Cacher les \switchocg{vals}{valeurs}/ \switchocg{units}{unités} 

\begin{table}[H]
    \centering
    \renewcommand{\arraystretch}{1.3} 
    \setlength{\tabcolsep}{10pt} 
    \arrayrulecolor{gray} 
    \rowcolors{2}{gray!15}{white} 
    \caption{Indices optiques de matériaux classiques}
    \begin{tabularx}{\linewidth-4ex}{@{} X X l @{}}
        \toprule
        \textbf{Matériau} & \textbf{Indice optique ($n$)} \\
        \midrule
        Vide & 1.0000 \\
        Air & 1.0003  \\
        Eau & 1.333  \\
        Verre (courant) & 1.5  \\
        Quartz & 1.46  \\
        Diamant & 2.42  \\
        \bottomrule
    \end{tabularx}
\end{table}

\begin{table}[H]
    \centering
    \renewcommand{\arraystretch}{1.3} 
    \setlength{\tabcolsep}{10pt} 
    \arrayrulecolor{gray} 
    \rowcolors{2}{gray!15}{white} 
    \caption{Plage de fréquences et longueurs d'onde du spectre visible et audible}
    \begin{tabularx}{\linewidth-4ex}{@{} X r r X @{}}
        \toprule
        \textbf{Domaine} & \textbf{Fréquence} & \textbf{Longueur d’onde} & \textbf{Unité} \\
        \midrule
        Lumière visible (violet) & \num{7.5e14} & \num{400} & \unit{\nano\meter} \\
        Lumière visible (rouge) & \num{4.3e14} & \num{700} & \unit{\nano\meter} \\
        Son audible (grave) & \num{20} & \num{17} & \unit{\meter} \\
        Son audible (aigu) & \num{20e3} & \num{1.7e-2} & \unit{\meter} \\
        \bottomrule
    \end{tabularx}
\end{table}

\begin{table}[H]
    \centering
    \renewcommand{\arraystretch}{1.3} 
    \setlength{\tabcolsep}{10pt} 
    \arrayrulecolor{gray} 
    \rowcolors{2}{gray!15}{white} 
    \caption{Champ magnétique et électrique terrestre}
    \begin{tabularx}{\linewidth-4ex}{@{} X r X @{}}
        \toprule
        \textbf{Paramètre} & \textbf{Valeur} & \textbf{Unité} \\
        \midrule
        Champ magnétique terrestre & \num{30} - \num{60} & \unit{\micro\tesla} \\
        Champ électrique terrestre (sol-air) & \num{100} - \num{300} & \unit{\volt\per\meter} \\
        \bottomrule
    \end{tabularx}
\end{table}

\begin{table}[h]
    \centering
    \renewcommand{\arraystretch}{1.3} % Espacement des lignes
    \setlength{\tabcolsep}{10pt} % Espacement des colonnes
    \arrayrulecolor{gray} % Couleur des lignes de tableau
    \rowcolors{2}{gray!15}{white} % Alternance des couleurs
    \caption{Caractéristiques des différentes sources lumineuses}
    \begin{tabularx}{\linewidth}{@{} X r c c c c X @{}}
        \hline
        \textbf{Source} & \(\lambda_m\) (nm) & \(\Delta\lambda\) (nm) & \(\tau_c\) (s) & \(l_c\) (m) & \(\tau_c / T_m\) \\
        \hline
        Laser rouge (TP) & \(\simeq 630\) & \(10^{-6} \text{ à } 10^{-3}\) & \(10^{-9} \text{ à } 10^{-6}\) & \(0,1 \text{ à } 1\) ``Un bras'' & \(10^7 \text{ à } 10^{10}\) \\
        \hline
        Raie verte du mercure & 546 & \(10^{-3} \text{ à } 1\) & \(10^{-12} \text{ à } 10^{-9}\) & \(10^{-4} \text{ à } 0,1\) & \(10^3 \text{ à } 10^6\) \\
        \hline
        Lumière blanche filtrée (filtre interférentiel) & \(\simeq 500\) & qques nm & \(10^{-13}\) & \(10^{-4}\) & \(\simeq 100\) \\
        \hline
        Lumière blanche filtrée (filtre gélatine) & \(\simeq 500\) & qques 10 nm & \(10^{-14}\) & \(10^{-5}\) & \(\simeq 10\) \\
        \hline
        Lumière blanche & \(\simeq 500\) & qques 100 nm & \(10^{-15}\) & \(10^{-6}\) & \(\simeq 1\) \\
        \hline
    \end{tabularx}
    \label{tab:source_lumineuse}
\end{table}

\end{document}
