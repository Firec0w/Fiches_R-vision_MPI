\documentclass[10pt,a4paper,twocolumn,landscape]{article}

\usepackage[utf8]{inputenc} % UTF-8 encoding
\usepackage[T1]{fontenc} % Proper font encoding for European languages
\usepackage[french]{babel} % French localization
\usepackage{microtype} % Improves text justification
\usepackage[scr=boondox]{mathalpha}
\usepackage{amsmath, amsfonts, amssymb}

\usepackage{tcolorbox}
\usepackage{fdsymbol}
\usepackage{float}
\usepackage{siunitx}

\usepackage{fancyhdr}
\usepackage{titling}

\usepackage{chemformula}

\begin{document}

\pagestyle{fancy}
\fancyhead{}

\title{Équations Redox : Fiche de révision}

\fancyhead[L]{Lycée Louis le Grand}
\fancyhead[C]{\thetitle}
\fancyhead[R]{Année 2024-2025}

\begin{center}{\large \bf Ajustage d'une demi-équation éléctronique}\end{center}

\begin{tcolorbox}[colback=white,colframe=black]
    \begin{enumerate}
        \item Assurer la conservation des éléments autres que $\ch{H}$ et  $\ch{O}$
        \item Assurer la conservation de l'élément en ajoutant de l'eau $\ch{H2O_{(l)}}$
        \item Assurer la conservation de l'élément hydrogène en ajoutant des ions $\ch{H^+_{(aq)}}$
        \item Assurer la conservation des charges en ajoutant des $e^-$
        \item Si le milieu est basique, ajouter autant d'$\ch{HO^-_{(aq)}}$ qu'il y a déjà d'$\ch{H^+_{(aq)}}$ et remplacer les paires $\ch{HO^-_{(aq)} + H^+_{(aq)}}$ par $\ch{H2O_{(l)}}$
    \end{enumerate}    
\end{tcolorbox}

\end{document}