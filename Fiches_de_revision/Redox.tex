\documentclass[10pt,a4paper]{article}

\usepackage[
    left=2cm, right=2cm, top=2cm, bottom=2cm,
    headheight=10mm, includehead
]{geometry}

\usepackage[utf8]{inputenc} % UTF-8 encoding
\usepackage[T1]{fontenc} % Proper font encoding for European languages
\usepackage[french]{babel} % French localization
\usepackage{microtype} % Improves text justification
\usepackage[scr=boondox]{mathalpha}
\usepackage{amsmath, amsfonts, amssymb}

\usepackage{tcolorbox}

\tcbset {
  base/.style={
    arc=0mm, 
    bottomtitle=0.5mm,
    boxrule=0mm,
    colbacktitle=black!10!white, 
    coltitle=black, 
    fonttitle=\bfseries, 
    left=2.5mm,
    leftrule=1mm,
    right=3.5mm,
    title={#1},
    toptitle=0.75mm, 
  }
}

\definecolor{brandblue}{rgb}{0.34, 0.7, 1}
\newtcolorbox{mainbox}[1]{
  colframe=brandblue, 
  base={#1}
}

\newtcolorbox{subbox}[1]{
  colframe=black!30!white,
  base={#1}
}

\usepackage{fdsymbol}
\usepackage{float}
\usepackage{siunitx}

\usepackage{fancyhdr}
\usepackage{titling}

\usepackage{chemformula}

\begin{document}

\pagestyle{fancy}
\fancyhead{}

\title{Équations Redox : Fiche de révision}

\fancyhead[L]{Lycée Louis le Grand}
\fancyhead[C]{\thetitle}
\fancyhead[R]{Année 2024-2025}

\begin{mainbox}{Définition}
    \begin{itemize}
        \item Un \textit{réducteur} est une espèce caable de \textit{céder} un ou plusieurs éléctrons
        \item Un \textit{oxidant} est un epèce capable de \textit{capter} un ou plusieurs éléctrons 
    \end{itemize}
\end{mainbox}

\begin{center}{\large \bf Ajustage d'une demi-équation éléctronique}\end{center}

\begin{subbox}{Ajustage d'une demi-équation éléctronique}
    \begin{enumerate}
        \item Assurer la conservation des éléments autres que $\ch{H}$ et  $\ch{O}$
        \item Assurer la conservation de l'élément en ajoutant de l'eau $\ch{H2O_{(l)}}$
        \item Assurer la conservation de l'élément hydrogène en ajoutant des ions $\ch{H^+_{(aq)}}$
        \item Assurer la conservation des charges en ajoutant des $e^-$
        \item Si le milieu est basique, ajouter autant d'$\ch{HO^-_{(aq)}}$ qu'il y a déjà d'$\ch{H^+_{(aq)}}$ et remplacer les paires $\ch{HO^-_{(aq)} + H^+_{(aq)}}$ par $\ch{H2O_{(l)}}$
    \end{enumerate}    
\end{subbox}

\end{document}